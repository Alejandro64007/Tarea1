\documentclass[12pt]{article}

% --- Página y tipografía ---
\usepackage[letterpaper,margin=2.5cm]{geometry}
\usepackage[T1]{fontenc}
\usepackage[utf8]{inputenc} % si compilas con pdfLaTeX
\usepackage{lmodern}
\usepackage{microtype}

% --- Imágenes y color ---
\usepackage{graphicx}
\usepackage{xcolor}

% --- Control fino de espacios ---
\usepackage{setspace}
\setlength{\parindent}{0pt}

\begin{document}
\thispagestyle{empty}

% ===== Encabezado con logos + texto =====
\begin{minipage}[c]{0.18\textwidth}
    \centering
    % Cambia por tu logo izquierdo
    \includegraphics[width=0.95\linewidth]{img/logo_usac.jpeg}
\end{minipage}
\hfill
\begin{minipage}[c]{0.60\textwidth}
    \small
    Universidad de San Carlos de Guatemala\\
    Escuela de Ciencias Físicas y Matemáticas\\
    Nombre estudiante\\
    Carnet: \\
    Programación 1\\
\end{minipage}
\hfill
\begin{minipage}[c]{0.18\textwidth}
    \centering
    % Cambia por tu logo derecho
    \includegraphics[width=1.4\linewidth]{img/logo_ecfm.jpg}
\end{minipage}

\vspace{0.5cm}

% Línea horizontal superior (gruesa)
\noindent\rule{\textwidth}{1.2pt}

\vspace{0.2cm}

% ===== Título =====
\begin{center}
    {\Large\scshape Titulo}\\[0.3em]
\end{center}

\vspace{0.1cm}

% Fecha
\begin{center}
    \small\scshape 29 de enero de 2026
\end{center}

\vspace{0.2cm}

% Línea horizontal inferior (gruesa)
\noindent\rule{\textwidth}{1.2pt}

\vspace{0.6cm}

% ===== Caja de resumen =====
\noindent
\colorbox{gray!35}{%
    \parbox{\textwidth}{%
        \vspace{0.6em}
        \textbf{Resumen}\\[0.3em]
        \small
        Prueba de funcionamiento de LaTex, No.10, Ensayo
        \vspace{0.8em}
    }%
}

La utilidad de Latex en el área de las Matemáticas y Físicas es bastante útil, el simple hecho de tener una
manera ordenada de poder estandarizar el texto de manera práctica y a conveniencia es increíble, siendo esto así,
espero poder usarlo más en el futuro, ya que es una herramienta realmente útil. 
\\
Desde siempre me ha gustado la ciencia, la física y el mundo en general, tratar de comprender todo aquello que
nos rodea, siempre sentí curiosidad sobre la formación de las estrellas, por qué se le llaman estrellas,
también sobre el clima y los diferentes fenómenos físicos que nos rodean, si pudiera especializarme en algo en concreto, sería 
en Geofísica; esto para entender el funcionamiento de la tierra y su ambiente físico; o bien, Astrofísica, 
ya que me encanta todo lo relacionado con el universo.
\\
Pero eso será para el futuro, ya que, hay que empezar a entender el mundo desde lo más básico para entender
cosas aún más complejas, asi que hay que esforzarse en la carrera :D.

\section{Objetivos}

\section{Marco Teórico}

\section{Diseño Experimental}

\section{Resultados}

\section{Discusión de Resultados}

\section{Conclusiones}

\section{Referencias}

\section{Anexos}
\end{document}

\end{document}
